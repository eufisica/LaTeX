\documentclass[12pt]{article}
\usepackage[portuguese]{babel}
\usepackage{amsfonts,amsmath,amssymb,gensymb}
\usepackage{array}
\usepackage{graphicx}
\usepackage{caption, subcaption}
\usepackage{tikz}
\usepackage{tkz-euclide}
\usepackage{circuitikz}
\usepackage{scalerel}
\usepackage{pict2e}
\usepackage{url}
\usepackage{hyperref}%onde quiser colocar o link usar o \url só para hiperligação ou \href 
\usepackage{fancyhdr}
\usepackage{float}
\usepackage{pgf}
\usepackage{pgfplots}
\usepackage{pgfplotstable}
\usepackage[utf8]{inputenc}
\pgfplotsset{compat=newest, width=10cm}
%\pgfplotsset{compat=1.18}
\usepackage{geometry}
\usepackage{wrapfig}
\usepackage{wrapstuff}
\usepackage{xcolor}
\usetikzlibrary{positioning, arrows.meta, intersections, calc, fadings, shapes.geometric, arrows, mindmap, backgrounds, shadings, decorations.markings, decorations.pathmorphing, math, shapes, decorations, patterns, shadows, external, angles, quotes, positioning, decorations.pathreplacing, shapes.misc}
\tikzset{>=latex} % for LaTeX arrow head
%\colorlet{myblue}{blue!80!black}
%\colorlet{myred}{black!50!red}
%\colorlet{glasscol}{blue!10}
%\tikzstyle{glass}=[top color=glasscol!90!black,bottom color=glasscol!90!black,middle color=glasscol,shading angle=40]
\usepackage{pgf}
\usepackage{pgf-spectra}
\usepackage{siunitx}
\sisetup{
	output-decimal-marker={,}% just uncomment if you want to use comma as the decimal marker!
}
\usepackage[shortconst]{physconst}
\usepackage{enumitem}
%\usepackage[inline]{enumitem}
\usepackage{tasks}
\usepackage{tabularx}
\usepackage{booktabs}
\usepackage{colortbl}
\usepackage{bohr}
\usepackage{tasks}
\usepackage{multicol}
\usepackage{multirow}
\usepackage{tcolorbox}
\usepackage{tikzorbital}
\usepackage{framed}
\usepackage{pstricks}% quando há ficheiros exportados do Inkscape compilar em XeLaTeX
%\usepackage{chemfig, chemmacros}
%\usepackage[inkscapeformat=png]{svg}
\usepackage{longtable, array}
\tcbuselibrary{raster}


\usepackage{calculator}
\usepackage{calculus}
\usepackage{calc}

% We will externalize the figures
\usepgfplotslibrary{external}
\usepgfplotslibrary{statistics}
\usepgfplotslibrary{fillbetween}
\tikzexternalize

%\usepackage[dvipsnames]{xcolor}

\usepackage{xparse}
\NewDocumentEnvironment{sidebyside}{O{.50} o +m +m}{%
	\noindent\begin{minipage}[t][][t]{#1\linewidth}%
		#3% Content of the first minipage
	\end{minipage}%
	\hfill%
	\noindent\begin{minipage}[t][][t]{\IfValueTF{#2}{#2}{#1}\linewidth}%
		#4% Content of the second minipage
	\end{minipage}\\% newline is important, it allows \hfill to work correctly, try removing it ;)
}

%caixa de texto olive e cyan
\newcommand{\ctoliv}[1]{\colorbox{olive!20}{\textbf{#1}}}
\newcommand{\ctcy}[1]{\colorbox{cyan!20}{\textbf{#1}}}

%setup texto e caixa colorida ao lado
\newcommand{\caixaladotexto}[2]{
	\begin{minipage}{0.55\linewidth}%
		#1
	\end{minipage}
	\hspace{0.05\linewidth}
	\begin{minipage}{0.40\linewidth}%
		\begin{tcbraster}[raster columns=1, raster equal height, colframe=orange!20, colback=orange!20]
			\begin{tcolorbox} 
				\vspace*{2pt}%Space before
				#2
				\vspace*{2pt}%Space after 
			\end{tcolorbox}
		\end{tcbraster}
	\end{minipage}%
}
\newcommand{\ladoalado}[2]{
	\begin{minipage}{0.45\linewidth}%
		\vspace*{2pt}%Space before
		#1
		\vspace*{2pt}%Space after
	\end{minipage}
	\hspace{0.05\linewidth}
	\begin{minipage}{0.45\linewidth}%
		\vspace*{2pt}%Space before
		#2
		\vspace*{2pt}%Space after
	\end{minipage}%
}

% Setup de texto colorido blue e olive
\newcommand{\toliv}[1]{\textcolor{olive!80}{\textbf{#1}}}
\newcommand{\tcy}[1]{\textcolor{cyan!80}{\textbf{#1}}}

%Setup caixa colorida a ocupar a linha
\newcommand{\caixacor}[1]{
	\vspace{15pt}
	\colorbox{olive!20}{%
		\begin{minipage}{\linewidth}%
			\vspace*{2pt}%Space before
			#1
			\vspace*{2pt}%Space after
		\end{minipage}% 
	}
}


\setlength\parindent{0pt} %to avoid indent
\graphicspath{ {./images/} } %pasta das imagens

\title{Lançamento Horizontal}
\author{José Gonçalves}

\begin{document}
	\maketitle
	
	\begin{figure}[H]
		\begin{tikzpicture}
			\coordinate (A) at (0,0);
			\coordinate (B) at (0,2);
			\coordinate (C) at (0,3);
			\coordinate (D) at (4,2);
			\coordinate (E) at (7,2);
			\coordinate (F) at (7,0);
			\coordinate (G) at (13,0);
			\draw (A)--(B)--(C)--(D)--(E)--(F);
			\draw[dashed] (B)--(D);
			\draw (F)--(G);
			\draw[<->, xshift=-6pt] (B)--(C) node[midway, anchor=east]{$h$};
			\draw[<->, xshift=-6pt] (F)--(E) node[midway, anchor=east]{$H$};
			\draw[dashed, ->] (F)--(7,4) node[anchor=east]{$y$};
			\draw[dashed, ->] (G)--(15,0) node[anchor=south]{$x$};
			\draw (0,3.2) circle [radius=0.2];
			\draw[dashed] (4,2.2) circle [radius=0.2];
			\draw[dashed] (7,2.2) circle [radius=0.2];
			\node[anchor=east] at (-0.3,3.3) {A};
			\node[anchor=east] at (4,1.7) {B};
			\node[anchor=east] at (6.9,1.7) {C};
			\node[anchor=south] at (12,0) {D};
			\node[anchor=west] at (-1.3,3.7) {$v_0(A)=0$};
			\draw[->, color=red] (7,2.2)--(7.6,2.2) node[midway, anchor=south]{$v_{x}$};
			\draw[dashed] (7.7,2.1) circle [radius=0.2];
			\draw[->, color=red] (7.7,2.1)--(8.3,2.1) node[midway, anchor=south]{$v_{x}$};
			\draw[->, color=red] (7.7,2.1)--(7.7,1.9) node[midway, anchor=east]{$v_{y}$};
			\draw[dashed] (8.4,1.9) circle [radius=0.2];
			\draw[->, color=red] (8.4,1.9)--(9.0,1.9) node[midway, anchor=south]{$v_{x}$};
			\draw[->, color=red] (8.4,1.9)--(8.4,1.4) node[midway, anchor=east]{$v_{y}$};
			\draw[dashed] (9.2,1.5) circle [radius=0.2];
			\draw[->, color=red] (9.2,1.5)--(9.8,1.5) node[midway, anchor=south]{$v_{x}$};
			\draw[->, color=red] (9.2,1.5)--(9.2,0.8) node[midway, anchor=east]{$v_{y}$};
			\draw[dashed] (10.1,0.8) circle [radius=0.2];
			\draw[->, color=red] (10.1,0.8)--(10.7,0.8) node[midway, anchor=south]{$v_{x}$};
			\draw[->, color=red] (10.1,0.8)--(10.1,-0.1) node[midway, anchor=east]{$v_{y}$};
			\draw[dashed] (10.5,0.2) circle [radius=0.2];
			\draw[->, color=red] (10.5,0.2)--(11.1,0.2) node[midway, anchor=south]{$v_{x}$};
			\draw[->, color=red] (10.5,0.2)--(10.5,-1.2) node[midway, anchor=east]{$v_{y}$};
			\node at (11.2,-0.2) {$x_{\text{máx}}$};
			\node at(6.8,-0.3){O};
		\end{tikzpicture}
	\end{figure}	
	
	\vspace{5mm}
	\section*{Movimento de A para C}
	Admitindo que no plano inclinado e no plano horizontal não existe força de atrito, podemos determinar a velocidade à saída da mesa pela conservação da energia mecânica. A esfera é largada de uma altura $h$, relativamente ao plano da mesa, e tem no ponto C energia cinética e energia devida à rotação da esfera, dependente do momento de inércia, $I$.
	
	\begin{align*}
		E_m(A) = E_m(B) &\Leftrightarrow mgh = \dfrac{1}{2} m v_C^2 + \dfrac{1}{2} I \omega^2 \Leftrightarrow \\ &\Leftrightarrow mgh = \dfrac{1}{2} m v_C^2 + \dfrac{1}{2} \cdot \dfrac{2}{5} m r^2 \cdot \dfrac{v_C^2}{r^2} \Leftrightarrow \\
		&\Leftrightarrow v_C = \sqrt{\dfrac{10}{7}gh}
	\end{align*}
	
	\vspace{5mm}
	\section*{Movimento de C para D}
	No movimento de lançamento horizontal, a esfera cai devido à aceleração da gravidade, $g$. Como tem velocidade inicial, na horizontal a sua velocidade mantém-se constante porque não existe nenhuma força a atuar nesta direção ($a_x=0$). Na vertical a sua velocidade irá aumentar, devido à aceleração da gravidade ($a_y=-g$).\\
	
	As condições iniciais são:\\
	\begin{center}
			$x_0 =0$ \hspace{2mm} e \hspace{2mm} $y_0=H$\\
		$v_{0x}=v_C=v_0$ e $v_{0y}=0$
	\end{center}

	As equações do movimento são:\\
	\begin{center}
		$x(t) = v_0 t$ \hspace{2mm} e \hspace{2mm} $y(t) = H - \dfrac{1}{2} g t^2$\\
		$v_x = v_0$  \hspace{2mm} e \hspace{2mm} $v_y = -gt$\\
		sendo $v = \sqrt{v_x^2 + v_y^2}$
	\end{center}
	
	O tempo de queda será quando a posição y(t) = 0. Assim, usando a equação da posição, temos:\\
		\begin{align*}
			y(t) = H - \dfrac{1}{2} g t^2 &\Leftrightarrow 0 = H - \dfrac{1}{2} g t^2 \Leftrightarrow \dfrac{1}{2} g t^2 = H \Leftrightarrow \\
			&\Leftrightarrow t^2 = \dfrac{2H}{g} \Leftrightarrow t_{\text{queda}} = \sqrt{\dfrac{2H}{g}}
		\end{align*}
	 
	 Substituindo este tempo de queda na equação da posição segundo $x$ temos o alcance máximo:\\ 
	 $$x_{\text{máx}} = v_0 \sqrt{\dfrac{2H}{g}}$$\\
	
	\section*{Reunindo a informação}
	A velocidade à saída da mesa é:\\
	\begin{equation}
		v_C = \sqrt{\dfrac{10}{7}gh}
		\label{eq:v0}
	\end{equation}
	
	O alcance máximo é dado por:\\
	\begin{equation}
		x_{\text{máx.}}=v_C\sqrt{\dfrac{2H}{g}}
		\label{eq:xmax}
	\end{equation}
	
	Substituindo a equação \ref{eq:v0} na velocidade inicial com que a esfera sai da mesa, $v_0$, na equação \ref{eq:xmax}, temos:
	\begin{equation}
		x_{\text{máx.}}=\sqrt{\dfrac{10}{7}gh}\cdot \sqrt{\dfrac{2H}{g}}
		\label{eq:xmax2}
	\end{equation}
	
	Simplificando, a equação \ref{eq:xmax2} fica:\\
	\begin{equation}
			x_{\text{máx.}} = \sqrt{\dfrac{20}{7} h \cdot H}
	\end{equation}
	
	\section*{Atividade experimental}
	Condições iniciais:\\
	Altura do plano inclinado à mesa - $h = \SI{0,145}{\meter}$\\
	Altura da mesa ao chão - $H = \SI{0,878}{\meter}$\\
	
	O alcance será:\\
	\def\hplano{0.145}
	\def\Hmesa{0.878}
	\DIVIDE{20}{7}{\tempA}
	\MULTIPLY{\hplano}{\Hmesa}{\tempB}
	\MULTIPLY{\tempA}{\tempB}{\tempC}
	\SQUAREROOT{\tempC}{\tempD}
	\def\xmax{\tempD}
	\begin{align*}
		x_{\text{máx.}} &= \sqrt{\dfrac{20}{7} h \cdot H}\\
		&= \sqrt{\tempA \cdot \hplano \cdot \Hmesa}\\
		&= \sqrt{\tempA \cdot \tempB} \\
		&= \sqrt{\tempC}\\
		&= \tempD \hspace{2pt} \unit{\meter}
	\end{align*}
	
	O valor obtido experimentalmente para o alcance da esfera foi de:\\
	$$x_\text{exp.} = \SI{51,3}{\centi\meter}$$
	
	O erro relativo foi de:\\
	\def\xexp{0.513}
	\SUBTRACT{\tempD}{\xexp}{\tempE}
	\DIVIDE{\tempE}{\tempD}{\tempF}
	\MULTIPLY{\tempF}{100}{\tempG}
	\TRUNCATE[1]{\tempG}{\sol}
	\def\erro{\sol}
	\begin{align*}
		\varepsilon_r &= \dfrac{|x_\text{máx.}-x_\text{exp.}|}{x_\text{máx.}} \times 100\\
		&=\dfrac{|\xmax-\xexp|}{\xmax} \times 100\\
		&=\erro \%
	\end{align*}
	
	\section*{Conclusão}
	Como o erro experimental foi significativo, podemos concluir que existe dissipação de energia durante a descida da esfera no plano inclinado e no plano horizontal.
\end{document}