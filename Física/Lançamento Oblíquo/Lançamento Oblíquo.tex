\documentclass{article}
\usepackage[portuguese]{babel}
\usepackage{amsfonts,amsmath,amssymb,gensymb}
\usepackage{array}
\usepackage{graphicx}
\usepackage{caption, subcaption}
\usepackage{tikz}
\usepackage{tkz-euclide}
\usepackage{circuitikz}
\usepackage{scalerel}
\usepackage{pict2e}
\usepackage{url}
\usepackage{hyperref}%onde quiser colocar o link usar o \url só para hiperligação ou \href 
\usepackage{fancyhdr}
\usepackage{float}
\usepackage{pgf}
\usepackage{pgfplots}
\usepackage{pgfplotstable}
\usepackage[utf8]{inputenc}
\pgfplotsset{compat=newest, width=10cm}
%\pgfplotsset{compat=1.18}
\usepackage{geometry}
\usepackage{wrapfig}
\usepackage{wrapstuff}
\usepackage{xcolor}
\usetikzlibrary{positioning, arrows.meta, intersections, calc, fadings, shapes.geometric, arrows, mindmap, backgrounds, shadings, decorations.markings,decorations.pathmorphing, math, shapes, decorations, patterns, shadows, external, angles, quotes, positioning}
\tikzset{>=latex} % for LaTeX arrow head
%\colorlet{myblue}{blue!80!black}
%\colorlet{myred}{black!50!red}
%\colorlet{glasscol}{blue!10}
%\tikzstyle{glass}=[top color=glasscol!90!black,bottom color=glasscol!90!black,middle color=glasscol,shading angle=40]
\usepackage{pgf}
\usepackage{pgf-spectra}
\usepackage{siunitx}
\sisetup{
	output-decimal-marker={,}% just uncomment if you want to use comma as the decimal marker!
}
\usepackage[shortconst]{physconst}
\usepackage{enumitem}
%\usepackage[inline]{enumitem}
\usepackage{tasks}
\usepackage{tabularx}
\usepackage{booktabs}
\usepackage{colortbl}
\usepackage{bohr}
\usepackage{tasks}
\usepackage{multicol}
\usepackage{multirow}
\usepackage{tcolorbox}
\usepackage{tikzorbital}
\usepackage{framed}
\usepackage{pstricks}% quando há ficheiros exportados do Inkscape compilar em XeLaTeX
\usepackage{chemfig, chemmacros}
\usepackage[inkscapeformat=png]{svg}
\usepackage{longtable, array}
\tcbuselibrary{raster}


% We will externalize the figures
\usepgfplotslibrary{external}
\usepgfplotslibrary{statistics}
\usepgfplotslibrary{fillbetween}
\tikzexternalize

%\usepackage[dvipsnames]{xcolor}

\usepackage{xparse}
\NewDocumentEnvironment{sidebyside}{O{.50} o +m +m}{%
	\noindent\begin{minipage}[t][][t]{#1\linewidth}%
		#3% Content of the first minipage
	\end{minipage}%
	\hfill%
	\noindent\begin{minipage}[t][][t]{\IfValueTF{#2}{#2}{#1}\linewidth}%
		#4% Content of the second minipage
	\end{minipage}\\% newline is important, it allows \hfill to work correctly, try removing it ;)
}

%caixa de texto olive e cyan
\newcommand{\ctoliv}[1]{\colorbox{olive!20}{\textbf{#1}}}
\newcommand{\ctcy}[1]{\colorbox{cyan!20}{\textbf{#1}}}

%setup texto e caixa colorida ao lado
\newcommand{\caixaladotexto}[2]{
	\begin{minipage}{0.55\linewidth}%
		#1
	\end{minipage}
	\hspace{0.05\linewidth}
	\begin{minipage}{0.40\linewidth}%
		\begin{tcbraster}[raster columns=1, raster equal height, colframe=orange!20, colback=orange!20]
			\begin{tcolorbox} 
				\vspace*{2pt}%Space before
				#2
				\vspace*{2pt}%Space after 
			\end{tcolorbox}
		\end{tcbraster}
	\end{minipage}%
}
\newcommand{\ladoalado}[2]{
	\begin{minipage}{0.45\linewidth}%
		\vspace*{2pt}%Space before
		#1
		\vspace*{2pt}%Space after
	\end{minipage}
	\hspace{0.05\linewidth}
	\begin{minipage}{0.45\linewidth}%
		\vspace*{2pt}%Space before
		#2
		\vspace*{2pt}%Space after
	\end{minipage}%
}

% Setup de texto colorido blue e olive
\newcommand{\toliv}[1]{\textcolor{olive!80}{\textbf{#1}}}
\newcommand{\tcy}[1]{\textcolor{cyan!80}{\textbf{#1}}}

%Setup caixa colorida a ocupar a linha
\newcommand{\caixacor}[1]{
	\vspace{15pt}
	\colorbox{olive!20}{%
		\begin{minipage}{\linewidth}%
			\vspace*{2pt}%Space before
			#1
			\vspace*{2pt}%Space after
		\end{minipage}% 
	}
}


\setlength\parindent{0pt} %to avoid indent
\graphicspath{ {./images/} } %pasta das imagens

\title{Lançamento Oblíquo}
\author{José Gonçalves}	






\begin{document}
	\maketitle
	
	\begin{tikzpicture}[scale=1.2,
		axis/.style={->, thick},
		vect/.style={->, blue!70!black, thick},
		point/.style={circle, fill=red, inner sep=1.8pt},
		dashedline/.style={dashed, black!60}
		]
		
		% ---------------------------------------------------------
		% AXES
		% ---------------------------------------------------------
		\draw[axis] (0,0) -- (11,0) node[right] {$x$};
		\draw[axis] (0,0) -- (0,5) node[above] {$y$};
		
		% Gravity arrow
		\draw[->, thick] (0.5,4.5) -- (0.5,3.5) node[right] {$\vec{g}$};
		
		% ---------------------------------------------------------
		% TRAJECTORY (parabolic)
		% ---------------------------------------------------------
		\draw[dotted, thick, orange!90!black, domain=0:10, samples=80]
		plot (\x, { -0.1*(\x-5)^2 +2.5 });
		
		% ---------------------------------------------------------
		% IMPORTANT POINTS ON THE TRAJECTORY
		% (approx. coordinates)
		% ---------------------------------------------------------
		\node[point] (P1) at (0,0) {};
		\node[point] (P2) at (2.8,2) {};
		\node[point] (P3) at (5,2.5) {};
		\node[point] (P4) at (8,1.6) {};
		\node[point] (P5) at (10,0) {};
		
		% ---------------------------------------------------------
		% VELOCITY VECTORS
		% ---------------------------------------------------------
		\draw[vect] (P1) -- ++(1,2) node[right] {$\vec{v}_0$};
		\draw[vect] (P1) -- ++(1,0) node[below right] {$\vec{v}_{0x}$};
		\draw[vect] (P1) -- ++(0,2) node[left] {$\vec{v}_{0y}$};
		
		\draw[vect] (P2) -- ++(1,0.5) node[right] {$\vec{v}$};
		\draw[vect] (P2) -- ++(1,0) node[below] {$\vec{v}_x = \vec{v}_{0x}$};
		\draw[vect] (P2) -- ++(0,0.5) node[right] {$\vec{v}_y$};
		
		\draw[vect] (P3) -- ++(1,0) node[right] {$\vec{v}_x = \vec{v}_{0x}$};
		\draw[vect] (P3) -- ++(0,0) node[above right] {$\vec{v}_y=0$};
		
		\draw[vect] (P4) -- ++(1,-0.5) node[right] {$\vec{v}$};
		\draw[vect] (P4) -- ++(1,0) node[below] {$\vec{v}_x = \vec{v}_{0x}$};
		\draw[vect] (P4) -- ++(0,-0.5) node[right] {$\vec{v}_y$};
		
		\draw[vect] (P5) -- ++(1,-2) node[below right] {$\vec{v}$};
		\draw[vect] (P5) -- ++(1,0) node[below] {$\vec{v}_x = \vec{v}_{0x}$};
		\draw[vect] (P5) -- ++(0,-2) node[right] {$\vec{v}_{0y}$};
		
		% ---------------------------------------------------------
		% LABELS FOR POINT NUMBERS
		% ---------------------------------------------------------
		\node[above left] at (P1) {1};
		\node[above left] at (P2) {2};
		\node[above left] at (P3) {3};
		\node[above left] at (P4) {4};
		\node[above left] at (P5) {5};
		
		% ---------------------------------------------------------
		% Horizontal dashed line for d_max
		
		\draw[dashedline] (0,-0.5) -- (10,-0.5) node[midway,below] {$d_{\max}$};
		
		% Vertical dashed line for H_max
		\draw[dashedline] (5,0) -- (5,2.5) node[midway,right] {$H_{\max}$};
		
	
		
		% ---------------------------------------------------------
		% Angle alpha at launch
		% ---------------------------------------------------------
		\draw (0.7,0) arc[start angle=0, end angle=65, radius=0.7];
		\node at (0.9,0.4) {$\alpha$};
		
	\end{tikzpicture}

\section*{Introdução}
	O lançamento oblíquo de um projétil é um tipo de lançamento em que a velocidade inicial é oblíqua, fazendo um certo ângulo, $\alpha$, com a direção horizontal. É o caso do movimento da bola da figura. Ao contrário do lançamento horizontal, neste lançamento a velocidade inicial tem duas componentes, uma horizontal e outra vertical. A figura mostra o esquema do movimento e respetiva imagem estroboscópica. A seguir escrevem-se as equações paramétricas do movimento no referencial indicado, assim como as componentes escalares da velocidade.\\
	
	\section*{Estudo do movimento segundo o sistema coordenado}
	\textbf{Condições iniciais}
	$$a_x=0 \hspace{5pt} \text{e} \hspace{5pt} a_y = -g$$ \\
	$$x_0 = 0 \hspace{5pt} \text{e} \hspace{5pt} y_0 = 0$$ \\
	$$v_{0x} = v_0 \cos \alpha \hspace{5pt} \text{e} \hspace{5pt} v_{0y} = v0 \sin \alpha$$ \\
	
	\textbf{Tipos de movimento e respetivas equações}\\
	Na direção horizontal, a componente da resultante das forças é nula. Logo, vamos ter um movimento uniforme nesta direção.\\
	Na direção vertical, a componente da resultante das forças não é nula, atuando sobre a bola a força da gravidade. Logo, vamos ter um movimento retardado nesta direção. \\
	As equações que regem o movimento segundo essas direções são:\\
	$$x(t) = v_0 \cos \alpha t \hspace{5pt} \text{e} \hspace{5pt} y(t) = v_0 \sin \alpha t - \frac{1}{2} g t^2$$\\
	$$v_x = v_0 \cos \alpha \hspace{5pt} \text{e} \hspace{5pt} v_y = v_0 \sin \alpha - gt$$\\
	sendo $v = \sqrt{v_x^2 + v_y^2}$\\
	
\section*{Altura máxima e alcance}
	Como mostra a figura, o módulo da componente escalar vertical da velocidade, $v_y$, diminui na subida e aumenta na descida, anulando-se no pontode altura máxima. Fazendo $v_y = 0$ na equação $v_y = v_0 \sin\alpha - gt$ obtém-se o tempo de subida do projétil:\\
	$$t_{\text{subida}} = \dfrac{v_0 \sin \alpha}{g}$$\\
	
	Consideremos o caso particular de um projétil que regressa ao plano horizontal de onde foi lançado. O tempo de subida é igual ao tempo de descida, pelo que o tempo de voo -- tempo que o projétil permanece no ar -- é o dobro do tempo de subida. Ou, fazendo $y(t) = 0$, obtém-se\\
	$$t_{\text{voo}} = 2 \dfrac{v_0 \sin \alpha}{g}$$\\
	
	Podemos determinar uma expressão para a altura máxima: substitui-se o tempo de subida, $t_{\text{subida}} = v_0 \sin \alpha g$ , na equação $y(t) = v_0 \sin \alpha t - \frac{1}{2} gt^2$,obtendo-se\\
	$$y_{\text{máx}} = \dfrac{v_0^2 \sin^2\alpha}{2g}$$\\
	
	Para determinar uma expressão para o alcance do projétil substitui-se o tempo de voo, $t_{\text{voo}} = 2 v_0 \sin \alpha g$, na equação $x(t) = v_0 \cos \alpha t$, e, tendo emconta que $\sin 2\alpha = 2 \sin\alpha \cos\alpha$, obtém-se\\
	$$x_{\text{máx}} = \dfrac{v_0^2 \sin 2\alpha}{g}$$\\
	
	As expressões anteriores permitem-nos tirar algumas conclusões para este tipo de projétil.\\
	 
	 \section*{Projétil lançado obliquamente que sai e regressa ao mesmo plano horizontal}
	 
	 
	 Para a mesma velocidade inicial, $v_0$:\\
	 \begin{figure}[H]
	 	\centering
	 	\begin{tikzpicture}[scale=1.2]
	 		
	 		% Eixos
	 		\draw[->, thick] (-0.2,0) -- (8,0) node[right] {$x$};
	 		\draw[->, thick] (0,-0.2) -- (0,4) node[above] {$y$};
	 		
	 		% Ângulos guia
	 		\draw[gray!40, thin] (0,0) -- ({3*cos(15)},{3*sin(15)});
	 		\draw[gray!40, thin] (0,0) -- ({3*cos(45)},{3*sin(45)});
	 		\draw[gray!40, thin] (0,0) -- ({3*cos(75)},{3*sin(75)});
	 		
	 		% Vetor 15°
	 		\draw[->, thick, red] (0,0) -- ({2.6*cos(15)},{2.6*sin(15)});
	 		\node[red] at (1.7,0.65) {$15^\circ$};
	 		
	 		% Vetor 45°
	 		\draw[->, thick, blue] (0,0) -- ({2.4*cos(45)},{2.4*sin(45)});
	 		\node[blue] at (1.1,1.4) {$45^\circ$};
	 		
	 		% Vetor 75°
	 		\draw[->, thick, green!70!black] (0,0) -- ({2*cos(75)},{2*sin(75)});
	 		\node[green!70!black] at (0.7,1.5) {$75^\circ$};
	 		
	 		% Trajetórias parabólicas aproximadas
	 		\draw[red!50] plot[domain=0:5] ({\x},{0.05*\x*(5-\x)});
	 		\draw[blue!70, thick] plot[domain=0:7] ({\x},{0.12*\x*(7-\x)});
	 		\draw[green!60!black, thick] plot[domain=0:4] ({\x},{0.3*\x*(4-\x)});
	 		
	 		% Linhas tracejadas de altura máxima
	 		\draw[dashed, green!60!black] (2,0) -- (2,1.2);
	 		\draw[dashed, blue!70] (3.5,0) -- (3.5,1.5);
	 		\draw[dashed, red!60!black] (2.6,0) -- (2.6,0.3);
	 	\end{tikzpicture}
	 \end{figure}
	 \begin{itemize}
	 	\item a altura máxima, $y_{\text{máx}} = \dfrac{v_0^2 \sin^2 \alpha}{2g}$, aumenta com o ângulo de lançamento $\alpha$;
	 	\item o tempo de voo, $t_{\text{voo}} = \dfrac{2v_0 \sin \alpha}{g}$, aumenta com o ângulo de lançamento $\alpha$.
	 \end{itemize}
	 
	 \begin{figure}[H]
	 	\centering
	 	\begin{tikzpicture}[scale=1.0]
	 		
	 		% Eixos
	 		\draw[->, thick] (-0.2,0) -- (10,0) node[right] {$x$};
	 		\draw[->, thick] (0,-0.2) -- (0,6) node[above] {$y$};
	 		
	 		% Estilo das curvas
	 		\tikzset{traj/.style={thick, blue!70}}
	 		
	 		% ----- Trajetórias -----
	 		% 20°
	 		\draw[traj] plot[domain=0:7] ({\x},{0.10*\x*(7-\x)});
	 		\node[blue!70] at (2.5,0.8) {$20^\circ$};
	 		
	 		% 30°
	 		\draw[traj] plot[domain=0:7.5] ({\x},{0.18*\x*(7.5-\x)});
	 		\node[blue!70] at (3.1,2) {$30^\circ$};
	 		
	 		% 45°
	 		\draw[traj] plot[domain=0:8] ({\x},{0.23*\x*(8-\x)});
	 		\node[blue!70] at (4.1,3.3) {$45^\circ$};
	 		
	 		% 60°
	 		\draw[traj] plot[domain=0:7.5] ({\x},{0.32*\x*(7.5-\x)});
	 		\node[blue!70] at (3.6,4.9) {$60^\circ$};
	 		
	 		% 70°
	 		\draw[traj] plot[domain=0:7] ({\x},{0.60*\x*(7-\x)});
	 		\node[blue!70] at (3.2,6.4) {$70^\circ$};
	 		
	 		% ----- Pontos de impacto -----
	 		\fill[blue!70] (7,0) circle (2.2pt);
	 		\fill[blue!70] (7.5,0) circle (2.2pt);
	 		\fill[blue!70] (8,0) circle (2.2pt);
	 		
	 	\end{tikzpicture}
	 	
	 \end{figure}
	 
	 O alcance, $x_{\text{máx}} = \dfrac{v_0^2 \sin 2\alpha}{g}$:\\
	 \begin{itemize}
	 	\item tem o valor máximo quando $\sin 2\alpha = 1 \Rightarrow 2\alpha = 90\degree \Rightarrow \alpha = 45\degree$,ou seja, para o ângulo de lançamento de \SI{45}{\degree};
	 	\item é igual para ângulos de lançamento complementares (isto é,cuja soma é \SI{90}{\degree}): por exemplo, \SI{20}{\degree} e \SI{70}{\degree}, ou \SI{30}{\degree} e \SI{60}{\degree}, pois o seno toma o mesmo valor.
	 \end{itemize}
	 
	 \textbf{NOTA:}
	 Quando os efeitos da resistência do ar não são desprezáveis, as trajetórias deixam de ser parabólicas e tanto o alcance como a altura máxima são inferiores aos correspondentes valores sem resistência do ar.Também o ângulo para o alcance máximo deixa de ser \SI{45}{\degree}: por exemplo,numa bola de golfe, para a qual o efeito da resistência do ar não é desprezável, o alcance máximo ocorre para o ângulo de lançamento de cerca de \SI{38}{\degree}.
	 
	 
\end{document}